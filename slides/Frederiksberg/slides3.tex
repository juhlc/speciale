\documentclass[10pt]{beamer}
\usepackage[danish]{babel}
 \usepackage[utf8]{inputenc} % til at skrive æÆøØåÅ.
\usepackage{amsmath,amssymb,amsfonts,amsthm,mathrsfs,latexsym}
\usepackage{tikz,tikz-cd}
\usepackage{diagbox} % til at lave diagonal i tabelcelle
\usepackage{graphicx}% til at rette størrelsen af tabel
\usepackage{mathtools}
\usepackage{rotating}
\usepackage[matrix,arrow,ps]{xy}
\usepackage{stmaryrd}
\usepackage{tcolorbox}
\newcommand*{\isoarrow}[1]{\arrow[#1,"\rotatebox{90}{\(\sim\)}"
]}
\usepackage[normalem]{ulem} % strikethrough on text: command \sout{text}

% Beamer setup
\usetheme[style=alternative,TPlrimage=kumat2
,ku]{Frederiksberg}

% Titlepage
\title{Elliptic Curves and Galois Representations}
\subtitle{Bachelor Defense}

\author[]{Rasmus Juhl Christensen\\
stud.scient.}
\institute[Faculty of Science]{Department of Mathematical Sciences}
\date[]{June 21st, 2021}

% Commands
\newcommand{\cupdot}{\mathbin{\mathaccent\cdot\cup}}

\newcommand{\mN}{\mathbb{N}}
\newcommand{\mZ}{\mathbb{Z}}
\newcommand{\mQ}{\mathbb{Q}}
\newcommand{\mR}{\mathbb{R}}
\newcommand{\mC}{\mathbb{C}}
\newcommand{\mP}{\mathbb{P}}
\newcommand{\mA}{\mathbb{A}}

\newcommand{\mF}{\mathbb{F}}

\newcommand{\mH}{\mathbb{H}}
\newcommand{\mD}{\mathbb{D}}

\newcommand{\cM}{\mathcal{M}}
\newcommand{\cH}{\mathcal{H}}
\newcommand{\cD}{\mathcal{D}}
\newcommand{\cF}{\mathcal{F}}
\newcommand{\cC}{\mathcal{C}}
\newcommand{\cSH}{\mathcal{SH}}

\newcommand{\M}{\text{M}}
\newcommand{\SL}{\text{SL}}
\newcommand{\GL}{\text{GL}}
\newcommand{\SO}{\text{SO}}
\newcommand{\PSL}{\text{PSL}}
\renewcommand{\d}{\hspace{1mm}d}
\newcommand{\p}{\partial}


\newcommand{\floor}[1]{\left\lfloor #1\right\rfloor}
\newcommand{\ceil}[1]{\left\lceil #1\right\rceil}
\newcommand{\leg}[2]{\left(\frac{#1}{#2}\right)}

\newcommand{\ths}{\hspace{0.5mm}}
\newcommand{\hs}{\hspace{1mm}}
\newcommand{\Hs}{\hspace{4mm}}
\newcommand{\HS}{\hspace{8mm}}

\newcommand{\abs}[1]{\left| #1\right|}
\newcommand{\pare}[1]{\left( #1\right)}
\newcommand{\bigPare}[1]{\bigg( #1\bigg)}
\newcommand{\cbrac}[1]{\left\{ #1\right\}}
\newcommand{\setBrac}[2]{\left\{ #1 \hs\middle|\hs #2 \right\}}
\newcommand{\brac}[1]{\left[ #1\right]}
\newcommand{\norm}[1]{\left\lVert#1\right\rVert}
\newcommand{\inProd}[1]{\left\langle#1\right\rangle}

\renewcommand{\Im}{\text{Im}\hs}
\renewcommand{\Re}{\text{Re}\hs}
\renewcommand{\phi}{\varphi}
\newcommand{\ord}{\text{ord}}
\newcommand{\supp}{\text{supp}\hs}
\newcommand{\Aut}[1]{\text{Aut}(#1)}
\newcommand{\stab}[2]{\text{Stab}_{#1}\pare{#2}}
\newcommand{\cusp}[1]{\text{Cusps}\pare{#1}}
\newcommand{\dist}[1]{\text{dist}\pare{#1}}

\newcommand{\nl}{\newline \\}
\DeclareMathOperator{\chara}{char}
\DeclareMathOperator{\Gal}{Gal}
\DeclareMathOperator{\End}{End}

\DeclareMathOperator{\Tr}{Tr}

\begin{document}
\frame[plain]{\titlepage}


\begin{frame}{Program}
\onslide<1->{
9.00-9.20: A presentation on the following topics:
\begin{enumerate}
	\item Central definitions: Elliptic curves and Galois representations
	\item Galois representations attached to elliptic curves
	\item Elliptic curves with complex multiplication
	\item Serre's open image theorem
	\item Elliptic curves defined over $\mQ$
	\item A criterion for surjectivity of adelic Galois representations attached to elliptic curves
	\item An example: \\
	A method to prove surjectivity of some $\ell$-adic Galois representations attached to elliptic curves
\end{enumerate}}
9.20: Questions.
\end{frame}

\begin{frame}{Elliptic curves}
Let $K$ be a perfect field. We define an elliptic curve in the following way:
\onslide<1->{
\begin{definition}[Elliptic curve]
Let $f(x,y)=y^2+a_1xy+a_3y-x^3-a_2x^2-a_4x-a_6\in K[x,y]$ satisfy that there exists no $P\in \bar{K}^2$, such that $f(P)=\frac{\partial f}{\partial x}(P)=\frac{\partial f}{\partial y}(P)= 0$. Then we say $f$ defines the affine elliptic curve $E/K$ by
$$E(\bar{K)}=\{P\in \bar{K}^2\mid f(P)=0\}$$
Let $F(X,Y,Z)=Z^3\cdot f(X/Z,Y/Z)\in K[X,Y,Z]$. Then the set 
\begin{align*}
&\{P\in \mP^2(\bar{K})\mid F(P)=0\}\\
=&\{[x_0:y_0:1]\in \mP^2(\bar{K})\mid (x_0,y_0)\in E(\bar{K})\}\cup \{[0:1:0]\}
\end{align*}
is the projective elliptic curve $E/K$. We define $\mathcal{O}\coloneqq [0:1:0]$
\end{definition}
}



\end{frame}




\begin{frame}{The group law}
Then we define the group law on elliptic curves:
\onslide<1->{\begin{theorem}
Let $E/K$ be an elliptic curve, $P=(x_1,y_1),Q=(x_2,y_2)\in E(K)$. 
If $x_1\neq x_2$, define $\lambda\coloneqq \frac{y_2-y_1}{x_2-x_1}$ and $\nu\coloneqq \frac{y_1x_2-y_2x_1}{x_2-x_1}$. If $P=Q$ and $\frac{\p f}{\p y}(P)\neq 0$, define $$\lambda\coloneqq -\frac{\frac{\p f}{\p x}(P)}{\frac{\p f}{\p y}(P)},\quad \nu\coloneqq y_1-\lambda x_1 $$ Then we let:
	\begin{align*}
	P+Q\coloneqq\big(&\lambda^2+a_1\lambda-a_2-2x_1,\\
	&-(\lambda+a_1)(\lambda^2+a_1\lambda-a_2-2x_1)-\nu-a_3\big)
\end{align*}
Else let $P+Q=\mathcal{O}$. Then $(E(K)\cup\{\mathcal{O}\},+)$ is an abelian group with $\mathcal{O}$ as the identity.
\end{theorem}
}

\end{frame}

\begin{frame}{Torsion points}
\onslide<1->{
\begin{definition}
Let $E/K$ be an elliptic curve, and let $n\in \mN$. Then for $P\in E(\bar{K})$, we define
$$nP\coloneqq \underbrace{P+\cdots +P}_{n\text{ times}}$$
and we define
$$E[n]\coloneqq \{P\in E(\bar{K})\mid nP=\mathcal{O}\}$$
\end{definition}}
\onslide<2->{
\begin{theorem}
Let $E/K$ be an elliptic curve, $n\in \mN$ coprime with $\chara K$ if $\chara K\neq 0$. Then
$$E[n]\cong(\mZ/n\mZ)^2$$
\end{theorem}
}
\end{frame}

\begin{frame}{Galois representations}
\onslide<1->{
\begin{definition}
Let $E/F$ be Galois with Galois group $\Gal(E/F)$, let $R$ be a topological ring and let $n\in \mN$. Then a Galois representation is  a continuous homomorphism $$\rho:\Gal(E/F)\to \GL_n(R)$$
\end{definition}
}

\onslide<2->{

\begin{definition}[Serre, McGill: Abelian $\ell$-adic Representations and Elliptic curves]
Let $\ell$ be a prime. If $V$ is a vector space over $\mQ_\ell$ of degree $n$, and $K$ is a field with separable algebraic closure $K_s$, then an $\ell$-adic representation of $G\coloneqq\Gal(K_s/K)$ is a continuous homomorphism
$$\rho : G\to \Aut{V}\left(\cong \GL_n(\mQ_\ell)\right)$$
\end{definition}
}

\end{frame}


\begin{frame}{Galois representations attached to elliptic curves}
Let $K$ be a perfect field, let $G_L=\Gal(\bar{K}/L)$ for $L/K$ an algebraic extension. Let $E/K$ be an elliptic curve and let $n\in \mN$ be coprime with $\chara K$ if $\chara K\neq 0$.\nl
\onslide<2->{ Now let $g\in G_K$ act on $P=(x_0,y_0)$ by $gP=(gx_0,gy_0)$, $g\mathcal{O}=\mathcal{O}$. From the group law, we get $g(mP)=mg(P)$, and so we get
$$\rho_{E,n}': G_K\to \Aut{E[n]}$$}
\onslide<3->{
Recall $E[n]\cong (\mZ/n\mZ)^2$. Hence, we can pick a basis $(P,Q)$ of $E[n]$ and get a mod $n$ Galois representation:
$$\rho_{E,n}: G_K\to \GL_2(\mZ/n\mZ)$$
}
\end{frame}

\begin{frame}Now fix a prime $\ell\neq \chara K$ if $\chara K\neq 0$. Picking a compatible system of bases, we can then draw the commutative diagram
\only<2-3>{
\begin{center}
\begin{tikzcd}[ampersand replacement=\&]
{}\&\Gal(\bar{K}/K) \arrow[two heads]{dl}{\rho_{E,\ell}'}\arrow[two heads]{d}{\rho_{E,\ell^2}'}\arrow[two heads]{dr}\& {}\& {}\\ 
\Aut{E[\ell]}\isoarrow{d} \&\Aut{E[\ell^2]} \arrow{l}{\mathrm{res}}\isoarrow{d}\& \cdots\arrow{l}\isoarrow{d} \& {}\\
\GL_2(\mZ/\ell\mZ) \&\GL_2(\mZ/\ell^2\mZ)\arrow{l}{\pmod{\ell}}\& \cdots\arrow{l}\& {}
\end{tikzcd}
\end{center}
}
\only<4->{
\begin{center}
\begin{tikzcd}[ampersand replacement=\&]
{}\&\Gal(\bar{K}/K) \arrow[two heads]{dl}{\rho_{E,\ell}'}\arrow[two heads]{d}{\rho_{E,\ell^2}'}\arrow[two heads]{dr} \arrow[two heads]{drr}{\rho_{E,\ell^\infty}'}\& {}\&{}\\ 
\Aut{E[\ell]}\isoarrow{d} \&\Aut{E[\ell^2]} \arrow{l}{\mathrm{res}}\isoarrow{d}\& \cdots\arrow{l}\isoarrow{d} \& \Aut{T_\ell[E]}\isoarrow{d}\\
\GL_2(\mZ/\ell\mZ) \&\GL_2(\mZ/\ell^2\mZ)\arrow{l}{\pmod{\ell}}\& \cdots\arrow{l}\&\GL_2(\mZ_\ell)
\end{tikzcd}
\end{center}
}
\onslide<3->{Taking inverse limits and defining $T_\ell[E]=\varprojlim_{n\in \mN}E[\ell^n]$, we get a $\ell$-adic Galois representation.}
\onslide<5->{If $\chara K=0$, then for a compatible system of bases, we may pack the $\ell$-adic Galois representations for all primes $\ell$ into a single adelic Galois representation via $\Aut{\prod_{p}T_p[E]}$:
$$\rho_{E}: G_K\to \GL_2(\hat{\mZ})$$}
\onslide<6->{These representations are only defined up to conjugation. Also, we can recover the mod $m$ and the $\ell$-adic Galois representation via projections $r_m:\GL_2(\hat{\mZ})\to \GL_2(\mZ/m\mZ)$, $\pi_{\ell}:\GL_2(\hat{\mZ})\to \GL_2(\mZ_\ell)$.}

\end{frame}
\begin{frame}{Complex multiplication}
Let $K$ be a perfect field, let $\ell$ be a prime such that $\ell\neq \chara K$ if $\chara K\neq 0$.
\onslide<2->{
\begin{definition}
Let $E/K$ be an elliptic curve. Then $\psi: E(\bar{K})\to E(\bar{K})$ is an endomorphism over $K$ if $\psi=[g_0:g_1:g_2]$ with $g_0,g_1,g_2\in K(E)$ regular (defined at each $P\in E(\bar{K})$) and $\psi(\mathcal{O})=\mathcal{O}$.
\end{definition}}
\onslide<3->{Any endomorphism $\psi$ of $E/K$ satisfies $$\psi(P+Q)=\psi(P)+\psi(Q)$$
for all $P,Q\in E(\bar{K})$.\\}
\onslide<4->{
An endomorphism over $K$ of an elliptic curve $E/K$ induces an endomorphism of $T_\ell[E]$ commuting with the endomorphisms induced by $G_K$. We have $\End(T_\ell[E])\cong M_2(\mZ_\ell)$, and so multiplication by $m$ maps for $m\in \mN$ are endomorphisms with action on $T_\ell[E]$ represented by $$\begin{pmatrix}
m&0\\0&m
\end{pmatrix}$$}


\end{frame}

\begin{frame}
\onslide<1->{
\begin{definition}
Let $E/K$ be an elliptic curve. If there exists an endomorphism of $E$ over $K$ that is not a multiplication by $m$ map, then we shall say $E$ has complex multiplication.

\end{definition}}


\onslide<2->{The following statements are then true:}
\onslide<3->{
\begin{theorem} If $E/K$ has complex multiplication, there exists a non-scalar matrix that commutes with $\rho_{E,\ell^\infty}(G_K)$ and $\rho_{E,\ell^\infty}(G_K)$ is abelian. Hence $\rho_{E,\ell^\infty}(G_K)$ is of infinite index in $\GL_2(\mZ_\ell)$ and in particular $\rho_{E,\ell^\infty}$ is not surjective.\nl}
\onslide<4->{If $E/K$ has complex multiplication over any field $L$ with $L/K$ a finite extension, then $\rho_{E,\ell^\infty}$ is then not surjective.\nl}
\onslide<5->{If $\chara K=0$, and $E/K$ has complex multiplication over any field $L$ with $L/K$ an algebraic extension, then $\rho_{E}$ is not surjective. Also, $\rho_E(G_K)$ is not open in $\GL_2(\hat{\mZ})$}
\end{theorem}
\end{frame}


\begin{frame}{Serre's open image theorem}
Let $K$ be a number field, $G_L=\Gal(\bar{K}/L)$ for $L/K$ an algebraic field extension.
\onslide<1->{

\begin{theorem}[Serre's open image theorem]
Let $E/K$ be an elliptic curve without complex multiplication. Then the image of $\rho_E$ in $\GL_2(\hat{\mZ})$ is open.
\end{theorem}

}
\onslide<2->{
This has the following equivalent formulations:
\begin{itemize}
	\item<3-> For all but finitely many primes $\ell$, $\rho_{E,\ell^\infty}$ is surjective.
	\item<4-> For all but finitely many primes $\ell$, $\rho_{E,\ell}$ is surjective.
	\item<5-> There exists $m\in \mN$, such that
	$$\{A\in \GL_2(\hat{\mZ})\mid r_m(A)=I\}\subseteq \rho_E(G_K)$$
	\item<6-> There exists $m\in \mN$, such that 
	$$\rho_E(G_K)=r_m^{-1}(\rho_{E,m}(G_K))$$
\end{itemize}
}

\end{frame}


\begin{frame}{Surjectivity of adelic Galois representations attached to elliptic cruves}
We can prove the following statements:
\begin{theorem}
Let $E/\mQ$ be an elliptic curve. Then $\rho_{E}$ is not surjective.
\end{theorem}
And the more general version:
 \begin{theorem}[Greicius 2010]
 Let $E/K$ be an elliptic curve over a number field $K$. Let $\Delta\in K^\times$ be the discriminant of any Weierstrass model of $E/K$. Then $\rho_E$ is surjective if and only if
\begin{enumerate}
 	\item the $\ell$-adic Galois representation $\rho_{\ell^\infty}: G_K\to \GL_2(\mZ_\ell)$ is surjective for all $\ell$,
 	\item $K\cap \mQ(\zeta_{\infty})=\mQ$ and
 	\item $\sqrt{\Delta}\not\in K(\zeta_\infty)$

 	\end{enumerate}
 \end{theorem}
\end{frame}


\begin{frame}{Determining if adelic Galois representations attached to elliptic curves are surjective}
We will be pursuing the following steps to find an elliptic curve with surjective adelic Galois representation:
\begin{enumerate}
	\item<2-> Determining if $K\cap \mQ(\zeta_{\infty})=\mQ$ and if $\sqrt{\Delta_E}\not\in K(\zeta_\infty)$.
	\item<3-> Reducing the set of primes, that could be \textit{exceptional}, to a finite set.
	\item<4-> Determining whether the $\ell$-adic Galois representations for each prime in the finite set is surjective.
\end{enumerate}
\onslide<5->{We will deal with the first two steps ad hoc. Reflecting the difficulty of the second step, we have the following unsolved problem:\nl }
\onslide<6->{
\textbf{Uniformity conjecture (Serre)}: \textit{For every number field $K$, there exists some prime $p$, such that for every elliptic curve $E/K$ and prime $\ell>p$, $\rho_{E,\ell^\infty}$ is surjective.\nl}
\onslide<5->{
For $K=\mQ$, $p=37$ is conjectured to be such a number.}}
\end{frame}

\begin{frame}{Surjectivity of $\ell$-adic Galois representations attached to elliptic curves}
Once again, let $K$ be a number field. \nl
\onslide<1->{We can reduce the final step to a manageable problem by the following theorem:
}
\onslide<2->{\begin{theorem}
Assume $\ell\geq 5$ is a prime, $E/K$ an elliptic curve and $\det:\rho_{E,\ell^\infty}(G_K)\to\mZ_\ell^\times$ is surjective. Then $\rho_{E,\ell}$ is surjective if and only if $\rho_{E,\ell^\infty}$ is surjective.\\
Further, $\rho_{E,8}$ surjective if and only if $\rho_{E,2^\infty}$ is surjective and $\rho_{E,9}$ is surjective if and only if $\rho_{E,3^\infty}$ is surjective.
\end{theorem}
}
\onslide<3->{Since the subgroups of $\GL_2(\mF_p)$ are classified stemming back to a book by Dickson from 1901, we may actually derive a sufficient condition for primes $p\geq 5$:
}

\end{frame}
\begin{frame}
\begin{theorem}
Let $\ell\geq 5$ and suppose $H\leq \GL_2(\mF_\ell)$ contains
\begin{enumerate}
	\item $s$ such that $\Tr(s)^2-4\det s$ is a non-zero square in $\mF_\ell$ and so that $\Tr(s)\neq 0$
	\item $s'$ such that $\Tr(s')^2-4\det s'$ is not a square in $\mF_\ell$ and so that $\Tr(s')\neq 0$
	\item $s''$ such that $u=\Tr(s'')^2/\det(s'')\neq 0,1,2,4$ and such that $u^2-3u+1\neq 0$
\end{enumerate}
Then $H$ contains $\SL_2(\mF_\ell)$. If further $\det: H\to \mF_\ell^\times$ is surjective, $H=\GL_2(\mF_\ell)$.
\end{theorem}
Choose a model of $E$ with coefficients in $\mathcal{O}_K$. 
\begin{theorem}
Let $\mathfrak{p}\nmid \ell$ be a prime ideal in $\mathcal{O}_K$ with $\Delta_E\not\equiv 0\pmod{\mathfrak{p}}$.  Let $t_\mathfrak{p}$ be the trace of the Frobenius map of $\tilde{E}_\mathfrak{p}$ (the reduction of $E$ modulo $\mathfrak{p}$) and $q_\mathfrak{p}$ the determinant. There exists $g\in G_K$ such that
$t_\mathfrak{p}\equiv \Tr\rho_{E,\ell}(g)\pmod{\ell},\quad  q_\mathfrak{p}\equiv \det\rho_{E,\ell}(g)\pmod{\ell}$
\end{theorem}
\end{frame}
\begin{frame}{An example}
We will now consider an elliptic curve with surjective adelic Galois representation.\onslide<2->{So let $\alpha$ be the real root of $x^3+x+1$, let $K=\mQ(\alpha)$ and let $E/K$ be the elliptic curve defined by $y^2+2xy+\alpha y=x^3 -x^2$.}
\onslide<3->{\begin{enumerate}
	\item A check shows that $K\cap \mQ(\zeta_\infty)=\mQ$ and $\sqrt{\Delta_E}\not\in K(\zeta_\infty)$.
	\item<4-> It is possible to prove that for any prime $\ell$ unramified in $K$, and prime ideal $\mathfrak{p}$ for which $\Delta_E\not \equiv 0\pmod{\mathfrak{p}}$ (and further $\ell\nmid v_\mathfrak{p}(j_E)$ if $\ell=2,3,5$) , then $\ell\mid 	\#\tilde{E}_\mathfrak{p}(\mathcal{O}_K/\mathfrak{p})$ if $\rho_{E,\ell}(G_K)\neq \GL_2(\mF_\ell)$. Using primes (2) and $(\alpha^2+\alpha+2)$, we get $\ell\mid 9,10$ if $\ell\neq 2,3,31$ and $\rho_{E,\ell}(G_K)\neq \GL_2(\mF_\ell)$.
	\item<5-> In the case $\ell=31$, we can find elements with traces and determinants satisfying the conditions listed needed to prove $\rho_{E,31}=\GL_2(\mF_31)$ as outlined in the previous slide. This can be done by using the method involving Frobenius maps at the primes $(7)$ and $(\alpha-2)$.}\\
	\onslide<6->{Similar but more technical arguments can be used to deal with the cases $\ell=2,3$.}
	
\end{enumerate}
\end{frame}


\begin{frame}{Questions}

\end{frame}
\begin{frame}{Questions}

\end{frame}
\begin{frame}{Questions}

\end{frame}



\end{document}